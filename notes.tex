\documentclass[12pt]{article}
\usepackage[utf8]{inputenc}
\usepackage[english]{babel}
\usepackage[margin=1in]{geometry}
\usepackage{graphicx}
\usepackage{hyperref}
\usepackage{amsmath}
\hypersetup{
    colorlinks,
    citecolor=black,
    filecolor=black,
    linkcolor=black,
    urlcolor=black
}

\title{CHE260: Heat Transfer}
\date{October 2021}

\begin{document}

\maketitle
\tableofcontents

\section{Introduction}

\begin{itemize}
    \item How is heat transfer different from thermodynamics? In thermodynamics, we assume quasi-equilibrium processes i.e. the time was not an important parameter. In heat transfer, time is an important parameter and we are interested in the rate of heat transfer. 
    \item What is the relationship between $\dot{Q}$ and $\Delta T$? What are the mechanisms of heat transfer? 
    \item Conduction: Transfer of heat through a medium that is stationary. 
    \item Convection: Transfer of heat from a solid surface and an adjacent fluid that is moving. Example: a fan blowing air over a hot plate. There is heat transfer from the hot plate into the fluid. 
    \item Radiation: Energy emitted by matter in the form of electromagnetic waves. 
    \item Radiation does not need a medium. In a vacuum, we can have radiation but not convection or condcution.
    \item Different mechanisms of heat transfer can take place simultaneously.
    \item Applications \begin{itemize}
        \item Power Generation \begin{itemize}
            \item Power plant: steam generation, condenser
            \item Automobiles: engine cooling, space heating/cooling
        \end{itemize}
        \item Buildings \begin{itemize}
            \item Heating / Cooling
            \item Hot water 
        \end{itemize}
        \item Refrigeration 
        \item Manufacturing \begin{itemize}
            \item Casting / Heat treatment
            \item  Injection Moulding
        \end{itemize}
    \end{itemize}
    

\end{itemize}


\section {Electronic Cooling}


\begin{itemize}
    \item $>$ 99 \% of the electrical energy supplied to a circuit is dissipated as heat
    \item Heat has to be dissipated to the environment while keeping the temperature of the chip in a certain range
    \item Heat is lost from the surface of the chip
    \item Important parameter is heat flux $ = \frac{\text{Heat Transfer Rate}}{\text{Unit Area}}$ (in $\frac{W}{\mathrm{cm}^2}$)
    \item To reduce heat flux, we can reduce heat generation and increase the surface area 
    \item As size increases, it becomes more difficult to lose heat
    \item Water cooling is more efficient for large systems compared to air cooling
\end{itemize}


\section{Radiation}

\begin{itemize}
    \item Radiation is energy emitted by all matter in the form of e.m. radiation
    \item Thermal radiation is emitted by all bodies at a finite temperature
    \item Opaque objects emit only from the surface
    \item Amount of radiation depends on the surface temperature. Summarized by the Stefan Boltzmann Law: $\dot{Q}_{emit} = \sigma A T_s^4$ where $\sigma$ is the Boltzmann constant ($5.67 \times 10^{-8}$ $\frac{W}{m^2k^4}$), $T_s$ is the surface temperature in Kelvin and $A$ is the surface area.
    \item A surface that emits as much radiation as this is called a "Blackbody". A real surface emits less than this.: $\dot{Q}_{emit} = \epsilon \sigma A T_s^4$ where $\epsilon$ is the emissivity and $0 \leq \epsilon \leq 1$ 
    \item Black paint has $\epsilon = 0.99$ which is very close to 1. Aluminum foil has a low emissivity of around 0.07. 
    \item If radiation is incident on a surface some will be absorbed. The fraction absorbed is a surface property known as the absorptivity $\alpha$ such that $\dot{Q}_{absorbed} = \alpha \cdot \dot{Q}_{incident}$ and $\dot{Q}_{reflected} = (1-\alpha) \cdot \dot{Q}_{incident}$
    \item Kirchoff's law says that $\alpha = \epsilon$
    \item Note: $\alpha$ and $\epsilon$ vary over different wavelengths
    \item Consider a special case of radiation \begin{itemize}
        \item Small surface which is completely surrounded by a much larger surface
        \item $T_s$, $A_s$ are temperature and area of the small surface (which is also the boundary), $T_{surr}$ is the temperature of the surrounding surface. Both surfaces are emitting and we are interested in the net emission
        \item $\dot{Q}_{rad} = \epsilon \sigma A_s (T_s^4 - T_{surr}^4)$ 
    \end{itemize}
    \item Example \begin{itemize}
        \item Chip with an area of $15 \times 15 mm$, $\epsilon = 0.6$, $T_{surr} = 25$. 
        \item Two methods of heat transfer \begin{itemize}
            \item Natural convection \begin{itemize}
                \item $h = c(T_s-T_{\infty})^{\frac{1}{4}}$
                \item $c = 4.2 \frac{W}{m^2 K ^{\frac{5}{4}}}$
                \item $q_{conv} = hA (T_S - T_{\infty})$
                \item $q_{rad} = \epsilon A (T_s ^4 - T_{surr}^4)$
            \end{itemize}
            \item Forced convetion: $h$ is constant at 250 $\frac{W}{m^2 K}$ \begin{itemize}
                \item $q_{conv} = h A (T_S - T_{\infty})$
            \end{itemize}
        \end{itemize}
    \end{itemize}
\end{itemize}

\section{Heat Conduction}
\begin{itemize}
    \item Heat Conduction Equation \begin{itemize}
    \item $x,y,z$ components of $\dot{Q}$
    \item $T$ is a function of $(x,y,z,t)$
    \item $\vec{\dot{Q}} = \dot{Q}_x \hat{i} + \dot{Q}_y \hat{j} + \dot{Q}_z \hat{k}$
    \item $\dot{Q}_x = -k A_x \frac{dT}{dx}$ (similar expressions for $\dot{Q}_y$ and $\dot{Q}_z$
    \end{itemize}
    \item One dimensional heat conductivity can model more complicated situations. For example if $\Delta x < < \Delta y, z$, $\frac{dT}{dx} >> \frac{dT}{dy},  \frac{dT}{dz}$ so that $\dot{Q}_y$ and $\dot{Q}_z$ can be neglected 
    \item One dimensional heat conduction \begin{itemize}
        \item Cross sectional area is $A(x)$ where $x$ is the coordinate along which heat transfer occurs
        \item $\dot{Q}_x$ at the entry and $\dot{Q}_{x + \Delta x}$ at the exit
        \item Want to find $T(x)$ inside the object
        \item Rate of increase of enthalpy $ = m c_p \frac{\partial T}{\partial t} = \rho V c_p \frac{\partial T}{\partial t} = \rho c_p A \Delta x \frac{\partial T}{\partial t}$
        \item Energy balance: \begin{itemize}
            \item $\rho c_p A \Delta x \frac{\partial T}{\partial t} = \dot{Q}_x - \dot{Q}_{x + \Delta x}$
            \item After simplifying and taking the limit as $\Delta x$ approaches 0, we get $\rho c_p \frac{\partial T}{\partial t} = \frac{-1}{A} \frac{\partial (\dot{q} A)}{\partial x}$
            \item $A$ depends on the coordinate system and we use Fourier's law for $\dot{q}$: $\dot{q} = -k \frac{dT}{dx}$        \end{itemize}
    \end{itemize}
    \item Cartesian Coordinates \begin{itemize}
        \item $A$ is a constant
        \item $p c_p \frac{\partial T}{\partial t} = \frac{\partial}{\partial x} [k \frac{\partial T}{\partial x}]$
        \item Assume $k$ is constant. Then $\frac{\partial T}{\partial t} = \alpha \frac{\partial^2 T}{\partial x^2}$ where $\alpha = \frac{k}{pc_p}$. 
        \item If steady state i.e. $\frac{\partial T}{\partial t} = 0$ then $\frac{d^2T}{d x^2} = 0$. 
        \item Units of $\alpha = \frac{k}{pc_p}$, the thermal diffusivity is $\frac{m^2}{s}$. 
        \item High $k$ means the material conducts well. High $pc_p$ means that the material stores energy
    \end{itemize}
    \item Cylindrical Coordinates \begin{itemize}
        \item Heat being conducted radially so $\dot{q} = - k \frac{\partial T}{\partial r}$ and $A = 2 \pi r L$
        \item $pc_p \frac{\partial T}{\partial t} = \frac{-1}{2 \pi r L} [\frac{\partial }{\partial r} \cdot \frac{\partial T}{\partial r}]$
        \item $\frac{1}{\alpha} \frac{\partial T}{\partial t} = \frac{1}{r} \frac{\partial }{\partial r}(r \cdot \frac{\partial T}{\partial r})$
        \item At steady state, $\frac{d}{dr} (r \frac{dT}{dr}) = 0$. 
    \end{itemize}
    \item Spherical Coordinates \begin{itemize}
        \item $A = 4 \pi r^2$ and $\dot{q} = -k \frac{\partial T}{\partial r}$ where $r$ is the radial spherical coordinate
        \item $\frac{1}{\alpha} \frac{\partial T}{\partial t} = \frac{1}{r^2} \frac{\partial}{\partial r}(r^2 \frac{\partial T}{\partial r})$
    \end{itemize}
    \item In general $\frac{1}{r^n} \frac{\partial}{\partial r} (r^n \frac{\partial T}{\partial r}) = \frac{1}{\alpha} \frac{\partial T}{\partial t}$ where 
    cartesian has $n = 0$, cylindrical has $n=1$ and spherical has $n = 2$. 
\end{itemize}

\section{Thermal Resistance}
\begin{itemize}
    \item At steady state, $\frac{d^2T}{dx^2} = 0$
    \item Heat flux: $\dot{q} = -k \frac{dT}{dx}$. 
    \item Heat flux is a constant
    \item Heat transer rate: $\dot{Q} = \dot{q} A = \frac{-kA(T_2-T_1)}{L}$
    \item $\dot{Q} = \frac{T_1 - T_2}{R_{wall}}$ where $T_1$ and $T_2$ are the temperatures of the walls
    \item $R_{cond} = R_{wall} = \frac{L}{kA}$
    \item Similar to current with voltage and Resistance
    \item $\dot{Q}_{conv} = hA (T_s - T_{\infty})$
    \item $R_{conv} = \frac{T_s-T_{\infty}}{Q_{conv}} = \frac{1}{hA}$
    \item Radiation is more complicated. $\dot{Q}_{rad} = \epsilon \sigma A(T_s^4 - T_{sur}^4)$
    \item We need to define a heat transfer coefficient for radiation. $h_{rad} = \frac{\epsilon \sigma A(T_s^4 - T_{sur}^4)}{A(T_s - T_{sur})}$
    \item $h_{rad} = \epsilon \sigma (T_s^2 + T_{sur}^2)(T_s + T_{sur})$
    \item Can treat it as a resistance. $R_{rad} = \frac{T_s - T_{sur}}{\dot{Q}_{rad}} = \frac{1}{h_{rad}A}$
    \item Multilayer Plane Wall \begin{itemize}
        \item Each layer has the same surface Area
        \item Layers have thicknesses $L_i$
        \item Temperature varies as $T_1$ on the outside, $T_2$, ..., $T_{n+1}$ where $n$ is the number of surfaces
        \item Treat each layer seperately as resistances in series
        \item For first wall $\dot{Q} = \frac{T_1-T_2}{R_1}$ so $T_1 - T_2 = \dot{Q}R_1$. In general, $T_i - T_{i+1} = \dot{Q}R_i$
        \item Summing all of them gives $T_1 - T_{n+1} = \dot{Q}(R_1 + ... + R_n)$ so that $\dot{Q} = \frac{T_1 - T_4}{R_{total}}$
        \item Therefore you can sum up resistances similar to electric circuits
        \item We can find each $R_i$ as $\frac{L_i}{k_i A}$
    \end{itemize}
    \item $\dot{Q} = UA(T_{\infty, 1} - T_{\infty, 4})$ where $U$ is the overall heat transfer
    \item Example: Refrigerator Wall \begin{itemize}
        \item 1 mm thick insulation on the outside and width of refrigerator is $L$. 
        \item $T_{room} = 25C$ and $T_{refrig} = 3C$
        \item $h_0 = 9 \frac{W}{m^2C}$ and $h_i = 4 \frac{W}{m^2C}$
        \item $k_{steel} = 15.1  \frac{W}{m^2C}$ and $k = 0.035 \frac{W}{m^2C}$ inside the refrigerator
        \item Constraint is that $T_{s, out} > 20$. We assume heat transfer from the outside room to the inside of the refrigerator
        \item What is $L$ to ensure $T_{s, out} > 20$ to prevent condensation on the outside of the refrigerator
        \item Thermal circuit consists of a convective resistance outside the refrigerator followed by 3 conductive resistance on the surfaces and then one convective resistance at the end
        \item INSERT PICTURE FROM LEC NOTES
        \item $\dot{Q} = \frac{T_{room} - T_{s,out}}{R_{conv, 0}} = \frac{T_{room} - T_{s, out}}{\frac{1}{h_0A}}$
        \item Consider unit area. Then $\dot{Q} = h_0 (T_{room} - T_{s, out}) = 9(25-20) = 45 W$
        \item $R_{total} = \frac{1}{h} + (\frac{L}{k})_{metal} + (\frac{L}{k})_{insulation} + (\frac{L}{k})_{metal} + \frac{1}{h_i} = \frac{1}{9} + \frac{10^{-3}}{15.1} + \frac{L_2}{0.035} + \frac{10^{-3}}{15.1} + \frac{1}{4} = 0.361 + \frac{L_2}{0.035}$
        \item $\dot{Q} = \frac{T_{room} - T_{refrig}}{R_{total}} \rightarrow 45 (0.361 + \frac{L_2}{0.035}) = 25 - 3$ so that $L = 45 mm$
    \end{itemize}
\end{itemize}
\section{Thermal Resistance Networks}
\begin{itemize}
    \item Multiple layers e.g. in an electric chip each with different thermal properties \begin{itemize}
        \item We are interested in $\dot{Q} = \dot{Q}_1 + \dot{Q}_2 + \dot{Q}_3 = \frac{T_1-T_2}{R_1} + \frac{T_1-T_2}{R_2} +  \frac{T_1-T_2}{R_3}$
        \item $ = (T_1-T_2)(\frac{1}{R_1} + \frac{1}{R_2} + \frac{1}{R_3}) = \frac{T_1-T_2}{R_{total}}$
        \item This is the electrical analog to parallel resistances
    \end{itemize}
    \item Thermal contact resistance \begin{itemize}
        \item So far we have been assuming perfect contact between different boundaries
        \item In reality, there is a rough surface at the boundary
        \item We can always define a thermal contact resistance $R_c = \frac{T_2-T_1}{\dot{q}}$ (units are $\frac{m^2 C}{W}$ (Note: Resistanc per unit area)
        \item The reciprocal of $R_C$ is known as the thermal contact conductance $h_c$. 
        \item $h = \frac{1}{R_c} = \frac{\dot{q}}{\Delta T}$ so that $\dot{q} = h_c \Delta T$
        \item $h_c$ is thus similar to the heat transfer coefficient
    \end{itemize}
    \item Heat conduction in cylinders \& spheres \begin{itemize}
        \item INSERT DRAWING FROM THE SLIDEs
        \item For a long pipe, the main temperature gradient is in the radial direction i.e. $\frac{dT}{dx} << \frac{dT}{dR}$
        \item Therefore we can assume 1-D radial conduction
        \item INSERT $r_1$, $r_2$ diagram 
        \item Solve heat conduction equation in cylindrical coordinates to get $T(r)$
        \item Steady state: $\frac{d}{dr}(r \frac{dT}{dr}) = 0$, at $r = r_1, T = T_1$ and at $r = r_2, T = T_2$
        \item Integrate this to get $T(r) = c_1 \ln r + c_2$
        \item Using the boundary conditions gives $c_1 = \frac{T_1-T_2}{\ln(\frac{r_1}{r_2})}$ and $c_2 = T_2 -  \frac{T_1-T_2}{\ln(\frac{r_1}{r_2})} \ln r_2$ 
    \end{itemize}
    \item Define thermal resistance of a cylinder \begin{itemize}
        \item $R_{cyl} = \frac{T_1-T_2}{\dot{Q}_{cond}} = \frac{\ln (\frac{r_2}{r_1})}{2 \pi L K}$
    \end{itemize}
\end{itemize}
\section{Conduction in cylinders and spheres, Insulation}
\begin{itemize}
    \item Inner temperature and then two surface layers
    \item $R_C = \frac{1}{hA} = \frac{1}{2 \pi r L h}$
    \item $R_{total} = R_{c,1} + R_{cond} + R_{c,2} = \frac{1}{2 \pi r_1 L h_1} + \frac{\ln (\frac{r_2}{r_!})}{2 \pi L R} + \frac{1}{2 \pi r_2 L h_2}$
    \item Insulation \begin{itemize}
        \item $R_{total} = R_{c, 1} + R_{cyl, 1} + R_{cyl, 2} + R_{c, 2}$ where $R_{cyl, 2}$ is for the insulation
        \item For a sphere \begin{itemize}
            \item Insulation around a spherical metal tank 
            \item $R_{total} = R_{c,1} + R_{sph, 1} + R_{sph, 2} + R_{c,2}$
            \item $= \frac{1}{4 \pi r_1^2 h_1} + \frac{r_2-r_1}{4 \pi r_1 r_2 k_1} + \frac{r_3 - r_2}{4 \pi r_3 r_2 k} + \frac{1}{4 \pi r_3^2 h_2}$
        \end{itemize}
    \end{itemize}
    \item R-value \begin{itemize}
        \item Thermal resistance. Thickness $L$, Surface area $A$ and thermal conductivity $k$
        \item $R = \frac{L}{k}$ is the R-value
        \item $\dot{Q} = \frac{\Delta T}{R} \times A$
        \item Units here are in imperial units i.e. $L$ is in feet, $k$ is in $\frac{Btu}{h ft F}$
    \end{itemize}
    \item Critical Radius of insulation \begin{itemize}
        \item Consider the area for heat loss
        \item Insulation: increase thickness, increasing conduction resistance and decreasing convective resistance 
        \item Can we increase heat transfer? \begin{itemize}
            \item Plot $\dot{Q}$ against $r_2$ to find the critical value of Resistance
            \item Equivalently set $\frac{d\dot{Q}}{dr_2} = 0$ to find the critical radius
        \end{itemize}
    \end{itemize}
\end{itemize}
\section {Heat Transfer from Finned Surfaces }
\begin{itemize}
    \item Read Chapter 17.6
    \item How to find $R_{heat sink}$ for finned surfaces e.g. heat sinks in computers
    \item Add diagram from notes: We have a cylinder with cross sectional area $A_c(x)$ and heat transfer coeff $h$. \begin{itemize}
        \item Consider a thin slice of this with thickness $\Delta x$ some distance $x$ away from the end
        \item Energy balance $\dot{Q}_{cond, x}$ in and $\dot{Q}_{cond, x+\Delta x}$ out
        \item Energy in = Energy out: $\dot{Q}_{cond, x} = \dot{Q}_{cond, x+\Delta x} + \dot{Q}_{conv}$
        \item Let the perimeter of fin be $P$. Then surface area of element is $P \Delta x$ so that $\dot{Q}_{conv} = h P \Delta x (T - T_{\infty})$
        \item Simplifying the energy balance by taking the limit as $\Delta x \rightarrow 0$: $\frac{d \dot{Q}_{cond}}{dx} + hP(T-T_{\infty}) = 0$
        \item Using Fourier's law $\dot{Q}_{cond} = -kA_C \frac{dT}{dx}$: $\frac{d}{dx} (kA_c \frac{dT}{dx}) - hP(T - T_{\infty}) = 0$
        \item Assuming $A_C, k, P$ constant: $\frac{d^2T}{dx^2} - \frac{hP}{kA_C}(T-T_{\infty}) = 0$
        \item Define $\Theta = T - T_{\infty}$ and $a^2 = \frac{hP}{kA_c}$ (constant) so that $\frac{d^2\Theta}{dx^2} - a^2 \Theta = 0$ where the solution is $\Theta(x) = c_1 e^{ax} + c_2 e^{-ax}$
        \item Boundary conditions: $T = T_b$ at the left end while on the right end as $L \rightarrow \infty$, $T = T_{\infty}$
        \item Therefore we simplify by having $T = T_{\infty}$ at $x = L$
        \item In terms of $\Theta$: At $x = 0$, $\Theta = T_b - T_{\infty} = \Theta_b$ and $\Theta (\infty) = 0$ 
        \item This gives $c_1 = 0$ and $c_2 = \Theta_b$ so that the solution is $\Theta(x) = \Theta_b e^{-ax}$ 
    \end{itemize}
    \item What is the heat loss from the fin? $\dot{Q}_b = -kA_c \frac{dT}{dx} |_{x = 0} \dot{Q}_{fin}$
    \item $\dot{Q}_{fin, long} = \sqrt{hPkA_C} (T_b - T_{\infty})$
    \item Finite fin length: What is the boundary condition at the open end \begin{itemize}
        \item We can heat transfer is negligible so that adiabatic and $\frac{dT}{dx} = 0$ at the boundary
        \item At $x = L$, $\frac{dT}{dx} = 0$ so that $\frac{d \Theta}{dx} = 0$: $c_1 e^{aL} - c_2 e^{-aL} = 0$ 
        \item At $x = 0$, $\Theta = \Theta_b$
        \item Solve for $c_1$ \& $c_2$ and the following solution will be obtained: $\frac{T(x) - T_{\infty}}{T_b - T_{\infty}} = \frac{\cosh a(L-x)}{\cosh aL}$ 
    \end{itemize}
    \item Can do the same for $\dot{Q}_{fin, insulated} = -k A_C \frac{dT}{dx} |_{x=0}$ which will give $\dot{Q}_{fin, insulated} = \sqrt{hPkA_c} (T_b - T_{\infty}) \tanh(aL)$ where $a = \sqrt{\frac{hP}{kA_c}}$
    \item To account for heat transfer from the tip, we can add a length $\Delta L$ at the end and the area there will be $A_c = \Delta L P$ ($P$ is the perimeter of the fin) so that the corrected length is $L_c = L + \frac{A_c}{P}$
\end{itemize}
\section{Heat transfer from finned surfaces (contd)}
\begin{itemize}
    \item Fin efficiency \begin{itemize}
        \item $\Delta T$ given by the difference between the fin temperature and the surrounding
        \item Most efficient fin would have a uniform temperature $T_b$ everywhere
        \item This would imply an infinite thermal conductivity
        \item In this case, $\dot{Q} = h A_{fin} (T_b - T_{\infty}) = hPL(T_b - T_{\infty})$
        \item We define $\eta_{fin} = \frac{dot{Q}_{fin}}{\dot{Q}_{fin, max}}$
        \item For an infinitely long fin, $\eta_{fin, long} = \frac{\sqrt{hPkA_c}(T_b - T_{\infty})}{hPL(T_b - T_{\infty})} = \frac{1}{L} \sqrt{\frac{kA_c}{hP}} = \frac{1}{aL}$ which is in terms of physical properties
        \item $\dot{Q}_{fin} = \eta_{fin} \dot{Q}_{fin, max} = \eta_{fin} hA_{fin} (T_b - T_{\infty}) = h \eta_{fin} A_{fin} (T_b-T_{\infty})$
        \item $\eta_{fin} A_{fin}$ can be treated as the corrected area
        \item For an insulated tip, perform the same steps with the original definition to get $\eta_{insulated tip} = \frac{\tanh{aL}}{aL}$
    \end{itemize}
    \item Fin effectiveness \begin{itemize}
        \item How much has the fin increased heat transfer by?
        \item $\epsilon_{fin} = \frac{\dot{Q}_{fin}}{\dot{Q}_{no fin}}$
        \item $\dot{Q}_{no fin} = hA_c(T_b - T_{\infty})$
        \item $\dot{Q}_{long fin} = \sqrt{hPkA_c} (T_b - T_{\infty})$ so that $\epsilon_{long fin} =  \sqrt{\frac{kP}{hA_c}}$
        \item To increase the effectiveness, make $k$ as large as possible and maximize $\frac{P}{A_c}$
        \item Fins are most effective with low $h$ so they are used for gases, hot liquids
        \item Generally we use fins if $\epsilon \geq 2$
    \end{itemize}
    \item When can we assume fins are infinitely long? \begin{itemize}
        \item $\frac{\dot{Q}_{fin, insulated}}{\dot{Q}_{fin, long}} = \tanh(aL)$. $\tanh$ asymptotically approaches 1 as $aL$ approaches $\infty$
        \item In practice, if $aL \geq 5$ we can assume an infinitely long fin. But even $aL = 1$ has $\tanh = 0.76$ so it gives 76 \% of heat transfer of an infinitely long fin. Therefore $L = \frac{1}{a}$ is a reasonable length for a fin 
    \end{itemize}
    \item Designing a heat sink \begin{itemize}
        \item $\dot{Q}_{total} = \dot{Q}_{unfinned} + \dot{Q}_{fin}$
        \item From the definition of efficience $\dot{Q}_{fin} = \eta_{fin} \cdot h A_{fin}(T_b - T_{\infty})$
        \item $\Rightarrow \dot{Q}_{total} = hA_{unfinned} (T_b-T_{\infty}) + h \eta_{fin} A_{fin} (T_b - T_{\infty}) = h \left[A_{unfinned} + \eta_{fin}A_{fin}\right] (T_b - T_{\infty})$
        \item We can define a thermal resistance $R_{fin} = \frac{T_b - T_{\infty}}{\dot{Q}_{total}} = \frac{1}{h \left[A_{unfinned} + \eta_{fin}A_{fin}\right]}$
    \end{itemize}
\end{itemize}
\section{Transient Heat Conduction} 
\begin{itemize}
    \item Consider a solid at temperture $T_i$ and a liquid at $T_{\infty} < T_i$. The solid is dropped into the liquid. How does $T$ vary over time? \begin{itemize}
        \item We would expect the temperature $T$ to asymptotically approach $T_{\infty}$
        \item Heat Conduction equation: $\frac{\partial^2 T}{\partial x^2} + \frac{\partial^2 T}{\partial y^2} + \frac{\partial^2 T}{\partial z^2} = \frac{1}{\alpha} \frac{\partial T}{\partial t}$
        \item We set the first three terms equal to 0 by using the lumped capacitance approximation i.e. no temperature gradient in the body
        \item We would expect this to be valid when the object is small and has a high thermal conductivity
        \item Using an energy balance: $\dot{E}_{store} = - \dot{Q}_{conv}$
        \item $\dot{E}_{store} = mc_p \frac{dT}{dt} = \rho V c_p \frac{dT}{d}$ and $\dot{Q}_{conv} = hA(T - T_{\infty})$
        \item Equating the two, we get $\frac{d(T - T_{\infty})}{T-T_{\infty}} = - \frac{hA}{\rho V c_p} dt$ so that $\ln (T - T_{\infty}) = \frac{-hA}{\rho V c_p} t + C_1$
        \item Using $T = T_i$ at $t = 0$, we get $\ln \left[ \frac{T-T_{\infty}}{T_i - T_{\infty}} \right] = \frac{-hA}{\rho V c_p} t$
        \item We define a "time constant" $\tau = \frac{\rho V c_p}{hA}$ so that $\frac{T-T_{\infty}}{T_i - T_{\infty}} = \exp \left[\frac{-t}{\tau}\right]$
        \item The LHS starts at 1 and decays to 0 as $t \rightarrow \infty$. Moreoever at $t  =\tau$, the value is $\frac{1}{e} \approx 0.368$ 
    \end{itemize}
    \item The response time of a thermometer is usually $3 \tau$. However it is important to note that $\tau$ is a function of $h$ so it varies in different environments
    \item Moreover $\tau$ depends on $\frac{V}{A} = \frac{r}{3}$ for a sphere so to get a fast response time you would make it very thin
    \item When is a lumped capacitance valid? \begin{itemize}
        \item At steady state, the conduction in a solid must be equal to the convection in a fluid i.e. $kA \frac{(T_1 - T_2)}{L} = hA (T_2 - T_{\infty})$ i.e. $\frac{T_1 - T_2}{T_2 - T_{\infty}} = \frac{hL}{k}$
        \item $\frac{hL}{k}$ is a dimensionless number and is known as a Biot number. 
        \item Note: $k$ is the thermal conductivity of the solid, $L$ is the length scale in the direction of conduction
        \item Suppose the Biot number is large i.e. $>> 1$ so that $T_1 - T_2 >> T_2 - T_{\infty}$
        \item If Biot number is very small, then $T_1 - T_2 << T_2 - T_{\infty}$ so we can neglect $T$ change inside the body (we assume uniform temp in the body) and use the lumped capacitance model
        \item By $<<$ we typically mean a Biot value $<0.1$
        \item In an irregular body, the length scale used is $L = \frac{V}{A}$
    \end{itemize}
\end{itemize}

\section {Transient Heat Conduction in 2 and 3 Dimensions}
\begin{itemize}
    \item A ball of volume $V$,mass $m$ and SA $A$ and heat transfer coefficient $h$ is droppped into a fluid
    \item Assume that the temperature $T$ is uniform in the body
    \item We had previously assumed that if $Bi < 0.1$ we have a lumped capacitance
    \item Example: Steel shaft, $k = 51.2$, $\rho = 7832$, $c = 541$ and $T_i = 300$ is placed into a furnace with $T_{\infty} = 1200$. \begin{itemize}
        \item How long before the shaft temperature reaches 800?
        \item We first calculate the Biot number as $Bi = \frac{hL}{k}$ where $L = \frac{V}{A} = \frac{\pi r^2 L}{2 \pi r L} = \frac{r}{2}$
        \item Then $Bi = \frac{h \frac{r}{2}}{k} = \frac{100 \times \frac{0.05}{2}}{51.2} = 0.05$ so we can apply the lumped value
        \item $\frac{T - T_{\infty}}{T_i - T_{\infty}} = \exp \left[ - \frac{hA}{\rho V c} t \right]$ where $\frac{A}{V} = \frac{2}{r}$
        \item This gives $\ln \left[ \frac{800-1200}{300-1200} \right] = \dots $ and solving gives $t = 859 s$
    \end{itemize}
    \item Transient heat conduction in 3 dimensions e.g. plane walls, cylinders, spheres \begin{itemize}
        \item What happens if $Bi > 0.1$? 
        \item In this case we cannot neglect the temperature gradients inside the body
        \item Have to solve the complete heat conduction equation
        \item Consider a solid wall with temperature $T_i$ on one side which then instantly becomes lowered to a temperature $T_{\infty}$ as it is placed into a fluid. 
        \item As time increases, the temperature inside the wall e.g. at the center decreases
        \item So in this case, $T$ is a function of $x$ and $t$
        \item For a plane wall $\frac{\partial^2 T}{\partial x^2} = \frac{1}{\alpha} \frac{\partial T}{\partial t}$ where $\alpha = \frac{k}{\rho c_p}$ is the thermal diffusivity
        \item Second order wrt $x$ so two boundary conditions are needed there. First order wrt $t$ so one initial condition is neeeded there
        \item This can be solved analytically but we will not do that
        \item We instead consider the lumped capacitance solution $\ln \left[ \frac{T-T_{\infty}}{T_i - T_{\infty}} \right] = \frac{-hA}{\rho V c_p} t$
        \item We take the characteristic length $L = \frac{V}{A}$. 
        \item $\frac{hA}{\rho V c_p} t = \frac{h}{\rho L c_p} t = (\frac{h}{\rho L c_p} t) (\frac{L}{L} \cdot \frac{k}{k}) = (\frac{hL}{k})(\frac{k}{\rho c_p})(\frac{t}{L^2}) = (\frac{hL}{k})(\frac{\alpha t}{L^2})$ where $Bi = \frac{hL}{k}$ which is unitless
        \item We define the Fourier number $Fo = \frac{\alpha t}{L^2}$ which is also dimensionless
        \item The dimensionless temperature $\Theta = \frac{T - T_{\infty}}{T_i - T_{\infty}}$ so that the lumped capacitance solution can be written as $\Theta = \exp (-Bi \cdot Fo)$
        \item A physical interpreation of the Fourier number can be found by considering a cube with side length $L$. Then $\dot{Q}_{cond} = k A \frac{\partial T}{\partial x} = k L^2 \frac{\Delta T}{L} = k L \Delta T$
        \item $\dot{Q}_{store} = m c_p \frac{\partial T}{\partial t} = \rho L^3 c_p \frac{\Delta T}{t}$ so that $\frac{\dot{Q}_{cond}}{\dot{Q}_{store}} = \frac{k}{\rho c_p} \cdot \frac{t}{L^2} = \frac{\alpha t}{L^2} = Fo$
        \item Even when we cannot assume lumped capacitance and get an exact solution of the heat conduction equation, the solution is of the form $\Theta = \Theta (Bi, Fo)$
        \item We can define $\Theta_0 = \frac{T_0 - T_{\infty}}{T_i - T_{\infty}}$ so that the solution is of the form $\Theta_0 = A_1 e^{-\lambda_1^2 Fo}$ where $A_1, \lambda_1$ are function of $Bi$
    \end{itemize}
    \item Example: Carbon steel plate with $T_i = 440$ is placed in a furnace at $T_{\infty} = 600$. We need to heat to a minimum temperature of 520. What is the time $t$ required. \begin{itemize}
        \item $h = 200$, $k = 40$ and $\alpha = 8 \times 10^{-6}$
        \item $Bi = \frac{hL}{k} = \frac{200 \times 0.04}{40} = 0.2$
        \item Since $Bi > 0.1$, we cannot use the lumped capacitance and instead use the analytical solution which we obtain from tables
        \item From Table 18.2, $Bi = 0.2 \implies \lambda_1 = 0.4328, A_1 = 1.0311$
        \item $\Theta_0 = \frac{T_0 - T_{\infty}}{T_i - T_{\infty}} = A_1 e^{- \lambda_1^2 Fo}$
        \item $\Theta_0 = \frac{520 - 600}{440-600} = 0.5$ so that $0.5 = 1.0311 \exp (- (0.4328)^2 Fo)$ and so $Fo = 3.864$
        \item $Fo = \frac{\alpha t}{L^2} \implies t = \frac{Fo L^2}{\alpha} = \frac{3.864 \times (0.04)^2}{8 \times 10^{-6}} = 773 s$
    \end{itemize}
\end{itemize}
\end{document}
